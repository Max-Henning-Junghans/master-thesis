% !TEX root = ../thesis.tex
%---------------------------------------------------------------------------
% Frontpage
%---------------------------------------------------------------------------

% Die Richtline zum Aufbau des Deckblatts von Bachelor- und Masterarbeiten
% findet sich hier:
% @see: http://www.uni-luebeck.de/fileadmin/uzl_ssc/PDF-Dateien/Richtlinie-Deckblatt-MINT-Abschlussarbeit-2012-10-18.pdf

\newcommand{\titlepageskip}{\vskip 20pt}

% @see: http://tex.stackexchange.com/questions/31705/different-margins-for-title-page
\newgeometry{top=1in,bottom=1in,right=1in,left=1.2in}
\begin{titlepage}

\title{Deutscher Titel der Bachelor-/Masterarbeit}
\author{Max Henning Junghans}

{\Large
	% 1. Offizielles Logo des Instituts, an dem die Arbeit angesiedelt ist. (Das offizielle Logo
	% enthält das Siegel der Universität zusammen mit dem Text "Universität zu Lübeck"
	% und darunter den Namen des Instituts.) Dieses Logo ist bei den Instituten zu
	% bekommen. Das Logo muss oben links platziert werden.
	\includegraphics[width=80mm]{Logo_Inst_Telematik_cropped}
	\vskip 44pt

	% 2. Optional: Noch einmal Name des Instituts und Angabe der Direktorin oder des
	% Direktors des Instituts.

	% 3. Titel der Arbeit in deutscher Sprache und ebenfalls in englischer Sprache. Dabei soll
	% die Sprache, in der die Arbeit verfasst wurde, als erste angeführt werden; die andere
	% Sprache kann weniger prominent dargestellt werden.
	% Auch bei englischsprachigen Studiengängen sollen die Titelblätter auf Deutsch sein.
	\textbf{\LARGE Konzeption und Entwicklung eines universellen Sensor- und Aktuatorsystems für Smart Gardening}\\
	\LARGE Design and Development of a Universal Sensor and Actuator System for Smart Gardening

	\titlepageskip
	% 4. Der Text "Bachelorarbeit" oder "Masterarbeit" (nicht "Bachelor-Arbeit" oder "Master-Arbeit").
	%\textbf{Bachelorarbeit}
	\textbf{Masterarbeit}

	\titlepageskip
	%5. Der Text "im Rahmen des Studiengangs"
	im Rahmen des Studiengangs\\
	%6. Der ausgeschriebene Name des Studiengangs (also beispielsweise "Informatik"
	%oder "Molecular Life Science", hingegen nicht "Bioinformatik" oder "MLS")
	\textbf{Informatik}\\
	%7. Der Text "der Universität zu Lübeck"
	der Universität zu Lübeck

	\titlepageskip
	%8. Der Text "Vorgelegt von" und der Name des Studenten
	vorgelegt von\\
	\textbf{Max Henning Junghans}

	\titlepageskip
	%9. Der Text "Ausgegeben und betreut von"
	ausgegeben und betreut von\\
	%10. Der Name des ersten Prüfers. Dies ist immer gleichzeitig
	%die Betreuerin oder der Betreuer im Sinne der Prüfungsordnung.
	\textbf{Dr.~rer.~nat.~Florian-Lennert~Lau}

	\vfill
	%13. Der Text "Lübeck, den" und das Abgabedatum.
	{
		Lübeck, den \abgabedatum
	}

	% Diesen Teil entfernen, wenn "Im Focus das Leben" nicht drauf stehen soll
	%14. Optional der Text "Im Focus das Leben".
	{
		\titlepageskip
		Im Focus das Leben
	}
}
\end{titlepage}
\restoregeometry

\cleardoublepage

% Erklärung
\newpage
\chapter*{Erklärung}

Ich versichere an Eides statt, die vorliegende Arbeit selbstständig und nur unter Benutzung
der angegebenen Hilfsmittel angefertigt zu haben.

\vspace*{3cm}
Lübeck, den \abgabedatum

\thispagestyle{empty}
\cleardoublepage


% Abstract und Kurzfassung


\section*{\huge Abstract}
Gardening is the most popular hobby among Germans, although not all tasks are enjoyable or carried out optimally, and there is therefore great potential for the use of smart gardening.
Demographic change and urbanisation also mean that fewer and fewer people have the time, opportunity, or ability to tend a garden.

The aim of this project is to develop a universal sensor and actor system for smart gardening that enables users to analyse, automate and optimise their garden according to their needs.
By collecting and analysing data, well-founded decisions can be made that can both increase yield and minimise resource consumption.
Many gardening tasks are repetitive, unexciting, and strenuous, which makes the possibility of automation particularly attractive.
Ultimately, gardening work can be optimised by automating tasks based on the data collected.
The target group is mainly amateur gardeners, which is why the system must be affordable and easy to use.

To fulfil these requirements, a concept for a modular and universal sensor and actuator system for smart gardening is being developed.
The system consists of a sensor and actuator case that can work autonomously and enables the universal connection of sensors and actuators as long as the corresponding interface for communication is implemented in the case.
Sensors and actuators can be created using a flexible definition format and controlled using self-definable rules.
In addition, a server is designed to provide extended functions such as notifications, the storage of sensor data and the integration of other data sources such as APIs.
A dashboard enables the user to visualise the collected data clearly.
This concept empowers the user to customise the system to their individual needs in the garden.

Furthermore, this concept is implemented as a prototype based on a Raspberry Pi and tested with a light sensor as well as a temperature, humidity, and air pressure sensor.
Weather data is also integrated into the system, simple notifications are realised, and the definition format is implemented in JSON.
Finally, a web application is being developed that includes the dashboard and enables the system to be configured.

\cleardoublepage


\section*{\huge Kurzfassung}
Gartenarbeit ist das beliebteste Hobby der Deutschen, wobei nicht alle Aufgaben Freude bereiten oder optimal ausgeführt werden und somit großes Potenzial für den Einsatz von Smart Gardening besteht.
Auch der demografische Wandel und die Urbanisierung führen dazu, dass immer weniger Menschen Zeit, die Möglichkeit oder die Fähigkeit haben, einen Garten zu pflegen.

Ziel dieser Arbeit ist es, ein universelles Sensor- und Aktuatorsystem für Smart Gardening zu entwickeln, das es dem Nutzer ermöglicht, den Garten nach eigenen Bedürfnissen zu analysieren, zu automatisieren und zu optimieren.
Durch die Sammlung und Analyse von Daten können fundierte Entscheidungen getroffen werden, die sowohl den Ertrag steigern, als auch den Ressourcenverbrauch minimieren können.
Viele Arbeiten im Garten sind repetitiv, wenig spannend und anstrengend, was die Möglichkeit zur Automatisierung besonders attraktiv macht.
Schlussendlich kann durch die Automatisierung von Arbeiten basierend auf den gesammelten Daten die Gartenarbeit optimiert werden.
Zielgruppe sind hauptsächlich Hobbygärtner, weshalb das System erschwinglich und einfach zu bedienen sein muss.

Um diesen Anforderungen gerecht zu werden, wird ein Konzept für ein modulares und universelles Sensor- und Aktuatorsystem für Smart Gardening konzipiert.
Das System besteht aus einem Sensor- und Aktuatorkoffer, der autark arbeiten kann und den universellen Anschluss von Sensoren und Aktuatoren ermöglicht, solange die entsprechende Schnittstelle für die Kommunikation im Koffer implementiert ist.
Über ein flexibles Definitionsformat können Sensoren und Aktuatoren angelegt und mithilfe von selbst definierbaren Regeln gesteuert werden.
Zusätzlich wird ein Server konzipiert, der erweiterte Funktionen wie Benachrichtigungen, die Speicherung von Sensordaten sowie die Integration weiterer Datenquellen wie APIs bereitstellt.
Ein Dashboard ermöglicht es dem Nutzer, die gesammelten Daten übersichtlich zu visualisieren.
Dieses Konzept ermächtigt den Nutzer, das System an seine individuellen Bedürfnisse im Garten anzupassen.

Weiterhin wird dieses Konzept als Prototyp umgesetzt, der auf einem Raspberry Pi basiert und mit einem Lichtsensor sowie einem Temperatur-, Luftfeuchtigkeit- und Luftdrucksensor getestet wurde.
Außerdem werden Wetterdaten in das System integriert, einfache Benachrichtigungen realisiert und das Definitionsformat in JSON umgesetzt.
Zuletzt wird eine Webanwendung entwickelt, die das Dashboard beinhaltet und die Konfiguration des Systems ermöglicht.

\cleardoublepage